\documentclass[a4paper,12pt]{article}
\pagestyle{plain}

%style
\usepackage[margin=1in]{geometry}
\usepackage{amsmath}
\usepackage{amsfonts}
\usepackage{amssymb}
\usepackage{setspace}    %\onehalfspace %\doublespace etc
\usepackage{fancyhdr}  % use this package to get a 2 line header
%\usepackage{times}          %times new roman
%\newsectionwidth{0pt}  % So the text is not indented under section headings




% Graphics and objects
\usepackage[pdftex]{color,graphicx}
\usepackage{pdfpages} %\includepdf[pages={-}]{myfile.pdf}

% Extra
%\usepackage{natbib}         %biblio
%\bibpunct{(}{)}{;}{a}{,}{,}   %biblio options


\usepackage{dsfont}
\newcommand{\E}{\mathbb{E}}
\newcommand{\R}{\mathbb{R}}
\newcommand{\Var}{\mathrm{Var}}
\newcommand{\Cov}{\mathrm{Cov}}
\newcommand{\PR}{\mathrm{Pr}}
\newcommand{\med}{\mathrm{med}}
\newcommand{\one}{\mathds{1}}
\newcommand{\e}{\varepsilon}



% Fancyhdr settings
\renewcommand{\headrulewidth}{0pt} % suppress line drawn by default by fancyhdr
\setlength{\headheight}{27.05003pt} % allow room for 2-line header
\setlength{\headsep}{24pt}  % space between header and text
\pagestyle{fancy}     % set pagestyle for document
\rhead{ {\it N. Frazier}\\{\it pg. \thepage} } % put text in header (right side)
\cfoot{}                                     % the foot is empty
\topmargin=-0.5in % start text higher on the page

% If you want a title page, ALT: see below
%\title{title\\ Here}
%\author{Nicholas Frazier}

\begin{document}
\vspace{-1cm}
{\large \noindent Notes \\    %%%%%%%
\normalsize N. Frazier \vspace{0cm} \\         %%%%%%%%%%%%%
\normalsize Rice University \vspace{.1cm} \\
\today
}
\thispagestyle{empty} % this page has no header  
\singlespace
%\vspace{-.8cm}
\normalsize
\hspace{.10in}





Data Generation
The individual's wage in the last period is given by 
        $$ln(w_{i,A}) = f_1 + f_2(x_{i,A}) + e_{i,A}.$$
Note, this does not include measurement error. For each individual the observed wage is
        $$ln(w_{i,A}) = f_1 + f_2(x_{i,A}) + e_{i,A}+v_{i,A}.$$
Calculation of Terminal Period Values
If it the final period of an individual's life, for a given wage $w_{i,A}$ and experience level $x_{i,a}$, the individual's value function is 
    $$V(w_{i,A},v_{i,A}) = max\{\gamma_1 + (1+\gamma_2)y_i,w_{i,A}+y_i\}.$$
Consequently, the individual chooses to work if $w_{i,A}+y_i \geq \gamma_1 + (1+\gamma_2)y_i$.

Calculation of Non-terminal Period Values
If it is not the final period of an individual's life, their value function when choosing not to work is
    $$V_a^0(x_{i,a}) = \gamma_1 + (1+\gamma_2)y_i + \left(\frac{1}{1+\delta}\right)V_{a+1}(x_{i,a}),$$
and their value function when choosing to work is 
    $$V_a^1(x_{i,a}) = w_{i,a}(x_{i,a}) + y_i + \left(\frac{1}{1+\delta}\right)V_{a+1}(x_{i,a}+1).$$
Thus, their value function at time $a$ with experience $x_{i,a}$ is 
    $$V_a(x_{i,a}) = max\{V_a^0(x_{i,a}), V_a^1(x_{i,a})\},$$
and the individual chooses to work if $V_a^1(x_{i,a}) > V_a^0(x_{i,a})$.

\section*{Summary Stats}

\begin{itemize}
	\item Individual lives for A periods
	\item
	\item 
	\item
	\item 
	\item
	\item 
	\item
\end{itemize}












\end{document}